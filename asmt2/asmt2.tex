\documentclass[11pt]{exam}
\usepackage{pdfsync}
\usepackage{amsmath}
\usepackage{amsthm}

%
%  Update these values for headers
%
\header{CS377\\Assignment 2 }{}{Colby Morrison}
\begin{document}

\begin{enumerate}
  \item[2.] A perfect interleaving gives a final value of $10$. The scenario table is divided by each iteration. We give the first 3 iterations, then continue in the same pattern 7 more times. 
    \begin{center}
    \begin{tabular}{|p{2cm}|p{2cm}|p{2cm}|p{2cm}|p{2cm}|}
      \hline
      \textbf{p} & \textbf{q} & \textbf{n} & \textbf{p.temp} & \textbf{q.temp}\\
      \hline
      \textbf{p1} & q1 & 0 & - & -\\
      p2 & \textbf{q1} & 0 & - & -\\
      \textbf{p2} & q2 & 0 & - & - \\
      p3 & \textbf{q2} & 0 & 0 & - \\
      \textbf{p3} & q3 & 0 & 0 & 0 \\
      p1 & \textbf{q3} & 1 & 0 & 0 \\ 
      \hline
      \textbf{p1} & q1 & 1 & 0 & 0\\
      p2 & \textbf{q1} & 1 & 0 & 0\\
      \textbf{p2} & q2 & 1 & 0 & 0 \\
      p3 & \textbf{q2} & 1 & 1 & 0 \\
      \textbf{p3} & q3 & 1 & 1 & 1 \\
      p1 & \textbf{q3} & 2 & 1 & 1 \\
      \hline
      \textbf{p1} & q1 & 2 & 1 & 1\\
      p2 & \textbf{q1} & 2 & 1 & 1\\
      \textbf{p2} & q2 & 2 & 1 & 1 \\
      p3 & \textbf{q2} & 2 & 2 & 1 \\
      \textbf{p3} & q3 & 2 & 2 & 2 \\
      p1 & \textbf{q3} & 3 & 2 & 2 \\
      \vdots & \vdots & \vdots & \vdots & \vdots\\
      p1 & \textbf{q3} & 10 & 9 & 9 \\
      \textbf{p1} & q1 & 10 & 9 & 9 \\
      (end) & \textbf{q1} & 10 & - & 9\\
      (end) & (end) & 10 & - & -\\
      \hline
          \end{tabular}
  \end{center}
\item[3.]
    \begin{tabular}{|p{2cm}|p{2cm}|p{2cm}|p{2cm}|p{2cm}|}
      \hline
      \textbf{p} & \textbf{q} & \textbf{n} & \textbf{p.temp} & \textbf{q.temp}\\
      \hline
      \textbf{p1} & q1 & 0 & - & -\\
      p2 & \textbf{q1} & 0 & - & -\\
      \textbf{p2} & q2 & 0 & - & - \\
      p3 & \textbf{q2} & 0 & 0 & - \\
      \textbf{p3} & q3 & 0 & 0 & 0 \\
      \hline
      \textbf{p1} & q3 & 1 & 0 & 0\\
      \textbf{p2} & q3 & 1 & 0 & 0\\
      \textbf{p3} & q3 & 1 & 1 & 0 \\
      \hline
      \textbf{p1} & q3 & 2 & 1 & 0\\
      \textbf{p2} & q3 & 2 & 1 & 0\\
      \textbf{p3} & q3 & 2 & 2 & 0 \\
      \hline
      \textbf{p1} & q3 & 3 & 2 & 0\\
      \textbf{p2} & q3 & 3 & 2 & 0\\
      \textbf{p3} & q3 & 3 & 3 & 0 \\
      \vdots & \vdots & \vdots & \vdots & \vdots\\
      \textbf{p1} & q3 & 8 & 7 & 0\\
      \textbf{p2} & q3 & 8 & 7 & 0\\
      \textbf{p3} & q3 & 8 & 8 & 0\\
      \hline
      \textbf{p1} & q3 & 9 & 8 & 0\\
      p2 & \textbf{q3} & 9 & 8 & 0\\
      p2 & \textbf{q1} & 1 & 8 & 0\\
      \textbf{p2} & q2 & 1 & 8 & 0\\
      p3 & \textbf{q2} & 1 & 1 & 0\\
      p3 & \textbf{q3} & 1 & 1 & 1\\
      \hline
      p3 & \textbf{q1} & 2 & 1 & 1\\
      p3 & \textbf{q2} & 2 & 1 & 1\\
      p3 & \textbf{q3} & 2 & 1 & 2\\
      \hline
      p3 & \textbf{q1} & 3 & 1 & 2\\
      p3 & \textbf{q2} & 3 & 1 & 2\\
      p3 & \textbf{q3} & 3 & 1 & 3\\
      \hline
      p3 & \textbf{q1} & 4 & 1 & 3\\
      p3 & \textbf{q2} & 4 & 1 & 3\\
      p3 & \textbf{q3} & 4 & 1 & 4\\
      \vdots & \vdots & \vdots & \vdots & \vdots\\
      p3 & \textbf{q3} & 9 & 1 & 9\\
      p3 & \textbf{q1} & 10 & 1 & 9\\
      \textbf{p3} & (end) & 10 & 1 & -\\
      \textbf{p1} & (end) & 2 & 1 & -\\
      (end) & (end) & 2 & - & - \\
      \hline
    \end{tabular}
  \item[4.] $n$ can range from $-k$ to $k$.
  \end{enumerate}
  \end{document}


